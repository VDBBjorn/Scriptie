\chapter{Inleiding}

\section{Bestaande situatie}
De enorme hoeveelheid informatie op het internet confronteert de eindgebruiker met het probleem van overaanbod.
Hoewel alle content beschikbaar is, is het voor de gebruiker moeilijk om de meest geschikte, de meest interessante
content terug te vinden. 

De klassieke sleutelwoord gebaseerde zoeksystemen bieden hierbij slechts een gedeeltelijke
oplossing. Voorkeuren van de gebruiker, eerdere ervaringen, en reeds geconsumeerde content worden niet in
rekening gebracht bij de klassieke zoeksystemen. Daarom zijn aanbevelingssystemen een nuttig hulpmiddel voor het
ontdekken, selecteren en consumeren van content. Vele online diensten zoals Amazon \cite{Everything2012} en Netflix \cite{Bennett2007} hebben
reeds een aanbevelingssysteem dat zijn nut al meermaals bewezen heeft. 

De klassieke aanbevelingssystemen werken op basis van consumptiegedrag (beoordelingen, aankopen, klikgedrag),
soms aangevuld met informatie over de content (metadata) zoals sleutelwoorden, auteur, categorie\"en, samenvatting,... Het probleem bij deze systemen is dat de beschrijving niet altijd perfect is. Zo kunnen er schrijffouten optreden, synoniemen of hyponiemen voorkomen, of een ander lexicon of thesaurus gebruikt zijn.

Aanbevelingssystemen zullen in dat geval niet goed werken, gezien deze een exacte overeenkomst van de metadata
velden verwachten of op zijn minst een overlap in gebruikte woorden. Zo zullen bij klassieke aanbevelingssystemen
twee synoniemen ten onrechte als twee verschillende concepten beschouwd worden. Bepaalde verbanden tussen
gerelateerde content items zullen dus nooit door een klassiek aanbevelingssysteem ontdekt worden.
Op het web is er echter meer informatie beschikbaar dan typisch gebruikt wordt in aanbevelingssystemen. Door het
“begrijpen” van de inhoud, kunnen aanbevelingssystemen meer nauwkeurige suggesties doen.

\section{Doelstelling}
Het doel van deze masterproef is de ontwikkeling van een aanbevelingssysteem die gebaseerd is op content
recognition in tekst. Via content recognition technieken kan informatie uit teksten gehaald worden die als input voor
een aanbevelingssysteem gebruikt kunnen worden. Informatie-extractie kan ervoor zorgen dat het
aanbevelingssysteem een beter beeld krijgt van de voorkeuren van de gebruiker door werkelijk rekening te houden
met de inhoudelijke aspecten van de items die aanbevolen worden. Tevens zal content recognition ervoor zorgen dat
de afhankelijkheid van metadata in aanbevelingssystemen minder sterk is, of zelfs verdwijnt.

\section{Overzicht}
TODO: hs1 -> ..., hs2 -> ...