%  Overzichtsbladzijde met samenvatting

\newpage

{
\setlength{\baselineskip}{14pt}
\setlength{\parindent}{0pt}
\setlength{\parskip}{8pt}

\begin{center}

\noindent \textbf{\huge
Verbeterde aanbevelingssystemen\\[8pt]op basis van content recognition in tekst
}

door 

Bjorn VANDENBUSSCHE

Scriptie ingediend tot het behalen van de academische graad van\\ 
Master of Science in de industri\"ele wetenschappen: informatica

Academiejaar 2015--2016

Promotor: Prof.~L.~MARTENS\\
Scriptiebegeleider: Dr.~ir.~T.~DE PESSEMIER\\

Faculteit Ingenieurswetenschappen en Architectuur\\
Universiteit Gent

Vakgroep Informatietechnologie

\end{center}

\section*{Samenvatting}

Het doel van dit onderzoek is om te achterhalen welke vormen van content recognition het aanbevelingssysteem kunnen helpen om nuttige aanbevelingen te leveren. Verschillende informatie-extractietechnieken kunnen ervoor zorgen dat het aanbevelingssysteem een beter beeld krijgt van de voorkeur van de gebruiker door werkelijk rekening te houden met de inhoudelijke aspecten van de items die aanbevolen worden.

Daartoe zijn verschillende vormen van content recognition uitgetest. Een eerst is classificatie. Door verschillende items binnen dezelfde categorie te classificeren kan de context al heel wat duidelijker afgeleid worden. Voor twee verschillende Nederlandse datasets werden met het WEKA framework de classifiers gezocht die de beste classificatie opleveren voor elke categorie.

Om vervolgens bepaalde \textit{topics} binnen categorie\"en of samenhorende items over verschillende categorie\"en heen te vinden wordt clustering toegepast met het k-means en fuzzy k-means algoritme in Apache Mahout. Ter optimalisatie van deze clustering wordt gebruik gemaakt van named entity recognition om een beter beeld te krijgen van topics.

Door zowel de informatie van de classificatie als van de clustering te combineren krijgt het content-based aanbevelingssysteem een beter beeld van de inhoud van items en worden betere aanbevelingen gegenereerd. 

\section*{Trefwoorden}

text mining, big data, aanbevelingssystemen, machine learning, classificatie, clustering, named entity recognition

}

\newpage % strikt noodzakelijk om een header op deze blz. te vermijden
